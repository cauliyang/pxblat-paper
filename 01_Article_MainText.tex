\leadauthor{Yangyang Li}

\title{PxBLAT: An Ergonomics and Efficient Python Binding Library for BLAT}
\shorttitle{Ergonomic Genomic Analysis with \gls{pxblat} }

\author[1]{Yangyang Li\orcidlink{0000-0001-8224-1067}}
\author[1,\Letter]{Rendong Yang \orcidlink {000-0003-0000-0000}}
\affil[1]{Department of Urology, Northwestern University Feinberg School of Medicine, Chicago, IL 60611}
% \affil[2]{B Institute, Chalk Road, Blackboardville, USA}
\date{}

\maketitle


\begin{abstract}
	\gls{pxblat} provides an ergonomic and efficient Python binding library for \gls{blat}, designed to enhance the user experience and performance of genomic analysis tasks.
	By providing a Pythonic interface to \gls{blat}, \gls{pxblat} simplifies the usage of \gls{blat}, allowing incorporating its functionality directly into Python-based bioinformatics workflows.
	Furthermore, A new parallelization and non-blocking model enables improved efficiency and intuitive user interface, thereby reducing the complexity of genomic data analysis tasks.
\end{abstract}


\begin{keywords}
	Software Libraries |  Sequence Analysis | BLAT
\end{keywords}

\begin{corrauthor}
	rendong.yang\at northwestern.edu
\end{corrauthor}

\section*{Introduction}\label{sec:introduction}

The continual advancement of genome sequencing technologies has led to an exponential increase in available genomic data.
Tools to analyze and manipulate these data have become critically important in both research and clinical contexts.
\gls{blat}~\citep{kent2002blat} is a standout in the bioinformatics field for its capability to perform rapid genome sequence alignments.
It offers a faster alternative to the \gls{blast}~\citep{altschul1990basic} for aligning DNA sequences to the human genome~\citep{kent2002blat}.
Despite its widespread use and acceptance in the bioinformatics community, interfacing with BLAT can present challenges.
\gls{blat} is implemented in the C language and provides a variety of utilities with a \gls{cli}, which is inconvenient for developing algorithms in Python.
So far, we lack a holistic approach to enhancing the usability of \gls{blat}.

The rise of Python as a favored programming language in bioinformatics is well documented due to its ease of use, extensive libraries, and versatility~\citep{perkel2015programming}.
Diverse libraries have been developed to extend Python's reach into other computing languages, improving the flexibility and interoperability of bioinformatics tools.
For instance, Biopython~\citep{cock2009biopython}, one of the most prominent bioinformatics libraries, provides interfaces to tools like \gls{blast}~\citep{altschul1990basic} and Clustal~\citep{higgins1988clustal}.
Nonetheless, to date, no comprehensive Python library for \gls{blat} has been developed.
Here we propose \gls{pxblat}, a Python library designed to streamline and enhance the interaction with \gls{blat}, thereby making it more efficient and ergonomic.
It enables one to employ \gls{blat} programmatically and integrate it  seamlessly within novel algorithms or analytical pipelines.
\gls{pxblat} serves as a bridge, bringing the high-performance capabilities of \gls{blat} into the Python environment while maintaining reproducibility.
Hence, \gls{pxblat} would open up new possibilities for efficient genomic analysis.
The overarching aim of this work is to fill the observed gap by providing a Python binding library specifically tailored for \gls{blat}, addressing both its efficiency and ergonomic concerns.

\section*{Implementation}\label{sec:implementation}

The design of \gls{pxblat} follows the principles of readability and simplicity, striving for an intuitive user experience that diminishes the learning curve for Python adepts.
In line with this, we have concentrated on minimizing the complexity while maximizing the usability and performance.
The original codebase of \gls{blat} is included by codebase of \gls{ucsc}.
We segregate the implementation of \gls{blat} from the codebase of \gls{ucsc} and minimize its dependencies.
Utilizing existing C codebase and reimplement utilities of \gls{blat} \(\left(\mathtt{V}37.1\right)\) suit  including \emph{faTwoBit}, \emph{gfServer} and \emph{gfClient} \gls{cli} using C\texttt{++} programming language~\citep{kent2002blat}.
Subsequently, we integrate the current C\texttt{++} code and develop \gls{pxblat} via Pybind11~\citep{pybind11}.
The approach facilitated direct interaction with the functions of the \gls{blat} without modifying its original source code, thus preserving the integrity and performance of \gls{blat} while broadening its capabilities.

The query result of \gls{pxblat} conforms to the \emph{QueryResult} class of Biopython~\citep{cock209biopython}, enabling  manipulation of the query result using Biopython features~\pCref{listing:example}.
Furthermore, \gls{pxblat} eliminates the need for intermediate files, allowing all operations to be executed in memory.
This eradicates the common obstacle of data conversion into specific formats, enabling users to focus more on the core sequence alignment task.
Input and output files have been made optional, offering more flexible choices.
The use of system calls, though functional, can induce latency and create performance bottlenecks.
\gls{pxblat} reduce the need of system calls, thereby improving efficiency~\pcref{tab:performance-evaluation}.
Moreover, \gls{pxblat} allows for the retrieval of status of server without manipulating log files, a process that can be troublesome in concurrent environments.
The library incorporates ergonomic features such as checking and waiting for server readiness for alignment, retrying different ports if current one is occupied, and using an existing server if one has already been started.

We provide a variety of examples and documententation to help user get started~\pcref{listing:example}.
Specifically, \gls{pxblat} provides \glspl{api} including Class \emph{Server}, \emph{Client} and other free functions to reproduce \gls{blat} suit.
Class \emph{Server} and \emph{Client} have same utilities as \gls{cli} \emph{gfServer} and \emph{gfClient} respectively but with greater flexibility.
Free functions, for example, \emph{start\_server}, \emph{query\_server}, \emph{status\_server}, \emph{fa2twobit}, \emph{twobit2fa} etc., serve as potential usage under different contexts.
The library has been tested and developed with \gls{ci} and \gls{cd} to ensure code quality.
It uses type annotation for public classes and functions to guarantee quality and correctness via type checker and static analyzer.
These type annotations also make \gls{pxblat} more user-friendly in developing environment, where the function signatures can be suggested and corrected automatically.
Besides, \gls{pxblat} contain \glspl{cli} implemented by its \gls{api}.
The \gls{cli} includes completion for different shells, improving versatility.

\begin{listing}
	\inputminted[linenos, breaklines]{python}{codes/example1.py}
	\mycaption{\gls{api} Example}{The code snippet shows how to use \gls{api} of \gls{pxblat},
		and the query result can be iterated. More code examples can be found at \url{https://pxblat.readthedocs.io/en}}
	\label{listing:example}
\end{listing}


\section*{Result}\label{sec:result}

The result and performance of \gls{pxblat} are benchmarked against \gls{blat} \(\left(\mathtt{V}37.1\right)\) based on dataset including \num{1000} FASTA Files.
The dataset, which consists of five groups,  is sampled from chromosome \num[round-mode=places, round-precision=0]{20} of the Human Genome (hg38), and each sample includes one sequence.
The dataset includes varying sizes to cover a range of typical usage scenarios.
The length of each sequence ranges from \num{1000} to \num{3000}~\pCref{suppfig:fas-len}.
The result benchmarking ensures the correctness of \gls{pxblat}~\pCref{supptab:cmp1}

\begin{table*}
	\centering
	\caption{Performance  Benchmarking}
	\label{tab:performance-evaluation}
	\begin{tabular}{lSSSS}
		\toprule
		Data Group      & {Samples (No.)} & {BLAT (Sec.)} & {PxBLAT (Sec.)} & {Speedup} \\
		\midrule
		benchmark/fas/1 & 200             & 61.4011       & 50.4143         & 1.2179    \\
		benchmark/fas/2 & 200             & 61.6994       & 51.0984         & 1.2075    \\
		benchmark/fas/3 & 200             & 55.3351       & 46.4404         & 1.1915    \\
		benchmark/fas/4 & 200             & 56.9760       & 47.7234         & 1.1939    \\
		benchmark/fas/5 & 200             & 54.5457       & 46.4391         & 1.1746    \\
		\bottomrule
	\end{tabular}
\end{table*}

The performance benchmarking is conducted on an Apple M1 Pro with macOS 13.4.1 22F82 arm64.
We use the system call to launch \gls{blat} and the \emph{time} library to measure the execution time.
Generally, \gls{pxblat} gains \textasciitilde\SI[per-mode=symbol,round-precision=0]{20}{\percent} speedup compared to \gls{blat}~\pCref{tab:performance-evaluation}.
In summary, \gls{pxblat}  offers tangible benefits according to reduced execution time and improved user experience, proving its value as an enhancement to BLAT's functionality.


\section*{Acknowledgements}\label{sec:acknowledgements}

Special thanks to the team who maintain \gls{ucsc} codebase and users from the bioinformatics community whose valuable feedback and suggestions were pivotal in refining \gls{pxblat}'s design and functionality;
We are grateful for the generous support from our institution and funding bodies, which made this project possible.
We look forward to continuing to contribute to its advancement.

\section*{Conflict of interest}\label{sec:conflict-of-interest}

None declared.

\section*{Funding}\label{sec:funding}


\section*{Data availability}\label{sec:data-availability}

The dataset for benchmarking and testing is available at the GitHub repository {\url{https://github.com/cauliyang/pxblat}.
The path of the dataset is \emph{benchmark/fas}.

\section*{Code availability}\label{sec:code-availability}

The \gls{pxblat}, along with the source code, is publicly available in the GitHub repository at \url{https://github.com/cauliyang/pxblat}.
The documentation is available at ReadtheDocs \url{https://pxblat.readthedocs.io/en/latest/}.
The script for benchmarking is available at \emph{tests/test\_result.py} in the repository.


\section*{Reference}\label{sec:reference}
\bibliographystyle{bxv_abbrvnat}
\bibliography{ref.bib}
