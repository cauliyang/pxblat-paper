\leadauthor{Yangyang Li}

\title{PxBLAT: An Efficient and Ergonomics Python Binding Library for BLAT}
\shorttitle{Ergonomic Genomic Analysis with PxBLAT}

\author[1]{Yangyang Li\orcidlink{0000-0001-8224-1067}}
% \author[2]{Second Doctor \orcidlink {000-0002-0000-0000}}
\author[1,\Letter]{Rendong Yang \orcidlink {000-0003-0000-0000}}
\affil[1]{Department of Urology, Northwestern University Feinberg School of Medicine, Chicago, IL 60611}
% \affil[2]{B Institute, Chalk Road, Blackboardville, USA}
\date{}

\maketitle

\begin{abstract}
	Abstract of the paper goes here.
\end{abstract}


\begin{keywords}
	Software Libraries |  Sequence Analysis | BLAT
\end{keywords}

\begin{corrauthor}
	rendong.yang\at northwestern.edu
\end{corrauthor}

\section*{Introduction}\label{sec:introduction}

This is introduction.

\section*{Materials and Methods}\label{sec:materials-and-methods}

This is materials and methods.
Pybind11 ~\citep{pybind11}

\subsection*{Design Philosophy of PxBLAT}\label{ssec:design-philosophy-of-pxblat}

% introduce features of pxblat
% 1. no intermidiate files, all in memory
% 2. no system call
% 3. no bother with log files to get status of tool
% 5. no need to worry about file format
% 6. no other dependency
% 4. higher proformance and Ergonomics

\subsection*{Implementation}\label{ssec:implementation}

% add generaal code usage example, introduce library api
% Server, Client
% and free function, start_server, stop_server, query_server, status_server, fa2twobit
% yet another command line application:
% pxblat server -h
% pxblat client -h
% pxblat fatotwobit -h
%

% introduce multiple thread version server and one thread one client model,

% query result will be parsed via BIO-python

\section*{Results}\label{sec:results}

\subsection*{PxBLAT has consistent result with BLAT}\label{ssec:pxblat-has-consistent-result-with-blat}

% show length distribution of fas data
% and show have same results

\subsection*{Benchmarking Performance}\label{ssec:benchmarking-performance}

% show performance with one thread one client model, compared to original command line version


\subsection*{Ergonomics}\label{ssec:ergonomics}


% add code example using with context
% with Server() as server:
%     client = Client()
%     client.start()
%     ret = client.get()
%
% show non-blocking client usage, and present its advantage in graph

% for example:
%
% server = Server()
% client = Client()
% client.start()
% do some stuffs that consuming time and do not need query result, then get query result
% result = client.get()
% do other stuffs that need query result
% close server when needs
% server.stop()



Text is added like this
This is a reference to a published paper \citep{watson_molecular_1953}.
We can cite other things too \citep{tipton_complexities_2019,zheng_genome_2011,alberts_molecular_2002}

This is a new paragraph.
New sentences on a new line.
New sentences on a new line.

% this is how to add a comment
This is a new result.
% this is how to add a figure with the name cells.
As you can see (Figure \ref{fig:cells}).

% full size figure is figure*
\begin{figure*}
	\centering
	\includegraphics[width=0.75\linewidth]{Figures/temp.png}
	\caption{\textbf{These are cells.}\\
		(\textbf{A}) This is a regular figure with a legend as a caption underneath. Inset: 3X zoom. Scale bar, \SI{10}{\micro\meter}.}
	\label{fig:cells}
\end{figure*}

It is possible to add a one-column Figure like this (Figure \ref{fig:nucleus}).

% one-column size figure is figure
\begin{figure}
	\centering
	\includegraphics[width=0.75\linewidth]{Figures/temp.png}
	\caption{\textbf{This is a nucleus.}\\
		(\textbf{A}) This is a one-column figure with a legend as a caption underneath.}
	\label{fig:nucleus}
\end{figure}


\section*{Discussion}\label{sec:discussion}

This is the discussion section where you wax lyrical about your findings.
You can put your work in the context of other published work \citep{brenner_uga:_1967}.

% discuss event-based io may be applied in the future.
%


\section*{Acknowledgements}\label{sec:acknowledgements}


\section*{Conflict of interest}\label{sec:conflict-of-interest}

\section*{Funding}\label{sec:funding}

\section*{Data availability}\label{sec:data-availability}


\section*{Reference}\label{sec:reference}

\bibliographystyle{bxv_abbrvnat}
\bibliography{refs.bib}






























